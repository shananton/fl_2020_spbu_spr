\documentclass[12pt]{article}
\usepackage[left=2cm,right=2cm,top=2cm,bottom=2cm,bindingoffset=0cm]{geometry}
\usepackage[utf8x]{inputenc}
\usepackage[english,russian]{babel}
\usepackage{cmap}
\usepackage{amssymb}
\usepackage{amsmath}
\usepackage{url}
\usepackage{pifont}
\usepackage{tikz}
\usepackage{verbatim}

\usetikzlibrary{shapes,arrows}
\usetikzlibrary{positioning,automata}
\tikzset{every state/.style={minimum size=0.2cm},
initial text={}
}


\newenvironment{myauto}[1][3]
{
  \begin{center}
    \begin{tikzpicture}[> = stealth,node distance=#1cm, on grid, very thick]
}
{
    \end{tikzpicture}
  \end{center}
}


\begin{document}
\begin{center} {\LARGE Формальные языки} \end{center}

\begin{center} \Large домашнее задание до 23:59 05.03 \end{center}
\bigskip

\begin{enumerate}
  \item Доказать или опровергнуть утверждение: произведение двух минимальных автоматов всегда дает минимальный автомат (рассмотреть случаи для пересечения, объединения и разности языков).

    Решение:

    Утверждение неверно во всех случаях. Более того, неверно даже то, что автомат
    будет минимальным с точностью до удаления недостижимых вершин.
    (в примере недостижимых вершин в произведении нет).

    Рассмотрим два языка над алфавитом $\{0, 1\}$.
    Пусть $L_1$ --- слова с нечетным количеством единиц,
    $L_2$ --- слова, заканчивающиеся на 1.

    Соответствующие минимальные автоматы:

    Для $L_1$:

    \begin{myauto}
      \node[state,initial]   (A)              {$A$};
      \node[state,accepting] (B) [right=of A] {$B$};
      \path[->]
        (A) edge [loop above]    node [above] {$0$} ()
            edge [bend right=20] node [below] {$1$} (B)
        (B) edge [loop above]    node [above] {$0$} ()
            edge [bend right=20] node [above] {$1$} (A)
      ;
    \end{myauto}

    Для $L_2$:

    \begin{myauto}
      \node[state,initial]   (X)              {$X$};
      \node[state,accepting] (Y) [right=of X] {$Y$};
      \path[->]
        (X) edge [loop above]    node [above] {$0$} ()
            edge [bend right=20] node [below] {$1$} (Y)
        (Y) edge [loop above]    node [above] {$1$} ()
            edge [bend right=20] node [above] {$0$} (X)
      ;
    \end{myauto}

    Автоматы для $L_1 \cap L_2$: автомат-произведение и минимальный соответственно:

    \begin{myauto}
      \node[state,initial]   (AX)               {$AX$};
      \node[state]           (AY) [right=of AX] {$AY$};
      \node[state,accepting] (BY) [below=of AX] {$BY$};
      \node[state]           (BX) [right=of BY] {$BX$};
      \path[->]
        (AX) edge [loop above]    node [above] {$0$} ()
             edge                 node [left]  {$1$} (BY)
        (AY) edge                 node [above] {$0$} (AX)
             edge [bend right=20] node [above] {$1$} (BY)
        (BX) edge                 node [right] {$1$} (AY)
             edge [loop below]    node [below] {$0$} ()
        (BY) edge                 node [below] {$0$} (BX)
             edge [bend right=20] node [below] {$1$} (AY)
      ;
    \end{myauto}

    \begin{myauto}
      \node[state,initial]   (AX/AY)                  {$AX/AY$};
      \node[state,accepting] (BY)    [below=of AX/AY] {$BY$};
      \node[state]           (BX)    [right=of BY]    {$BX$};
      \path[->]
        (AX/AY) edge [loop above]    node [above] {$0$} ()
                edge [bend right=20] node [left]  {$1$} (BY)
        (BY)    edge [bend right=20] node [right] {$1$} (AX/AY)
                edge                 node [below] {$0$} (BX)
        (BX)    edge [loop below]    node [below] {$0$} ()
                edge                 node [above] {$1$} (AX/AY)
      ;
    \end{myauto}

    Автоматы для $L_1 \cup L_2$: автомат-произведение и минимальный соответственно:

    \begin{myauto}
      \node[state,initial]   (AX)               {$AX$};
      \node[state,accepting] (AY) [right=of AX] {$AY$};
      \node[state,accepting] (BY) [below=of AX] {$BY$};
      \node[state,accepting] (BX) [right=of BY] {$BX$};
      \path[->]
        (AX) edge [loop above]    node [above] {$0$} ()
             edge                 node [left]  {$1$} (BY)
        (AY) edge                 node [above] {$0$} (AX)
             edge [bend right=20] node [above] {$1$} (BY)
        (BX) edge                 node [right] {$1$} (AY)
             edge [loop below]    node [below] {$0$} ()
        (BY) edge                 node [below] {$0$} (BX)
             edge [bend right=20] node [below] {$1$} (AY)
      ;
    \end{myauto}

    \begin{myauto}
      \node[state,accepting] (BX/BY)                  {$BX/BY$};
      \node[state,initial]   (AX)    [above=of BX/BY] {$AX$};
      \node[state,accepting] (AY)    [right= of AX]   {$AY$};
      \path[->]
        (BX/BY) edge [loop below]    node [below] {$0$} ()
                edge [bend right=20] node [right] {$1$} (AY)
        (AY)    edge [bend right=20] node [left]  {$1$} (BX/BY)
                edge                 node [above] {$0$} (AX)
        (AX)    edge [loop above]    node [above] {$0$} ()
                edge                 node [left]  {$1$} (BX/BY)
      ;
    \end{myauto}

    Автоматы для $L_1 \setminus L_2$: автомат-произведение и минимальный соответственно:

    \begin{myauto}
      \node[state,initial]   (AX)               {$AX$};
      \node[state]           (AY) [right=of AX] {$AY$};
      \node[state]           (BY) [below=of AX] {$BY$};
      \node[state,accepting] (BX) [right=of BY] {$BX$};
      \path[->]
        (AX) edge [loop above]    node [above] {$0$} ()
             edge                 node [left]  {$1$} (BY)
        (AY) edge                 node [above] {$0$} (AX)
             edge [bend right=20] node [above] {$1$} (BY)
        (BX) edge                 node [right] {$1$} (AY)
             edge [loop below]    node [below] {$0$} ()
        (BY) edge                 node [below] {$0$} (BX)
             edge [bend right=20] node [below] {$1$} (AY)
      ;
    \end{myauto}

    \begin{myauto}
      \node[state,initial]   (AX/AY)                  {$AX/AY$};
      \node[state]           (BY)    [below=of AX/AY] {$BY$};
      \node[state,accepting] (BX)    [right=of BY]    {$BX$};
      \path[->]
        (AX/AY) edge [loop above]    node [above] {$0$} ()
                edge [bend right=20] node [left]  {$1$} (BY)
        (BY)    edge [bend right=20] node [right] {$1$} (AX/AY)
                edge                 node [below] {$0$} (BX)
        (BX)    edge [loop below]    node [below] {$0$} ()
                edge                 node [above] {$1$} (AX/AY)
      ;
    \end{myauto}
  \item Для регулярного выражения:
   \[ (a \mid b)^+ (aa \mid bb \mid abab \mid baba)^* (a \mid b)^+\]
  Построить эквивалентные:
  \begin{enumerate}
    \item Недетерминированный конечный автомат
    \item Недетерминированный конечный автомат без $\varepsilon$-переходов
    \item Минимальный полный детерминированный конечный автомат
  \end{enumerate}

    Решение:

    Заметим, что это регулярное выражение принимает ровно те слова, в которых хотя
    бы 2 символа (a или b).

    Доказательство.

    Пусть слово принимается выражением. Тогда хотя бы один символ соответствует
    первой скобке, и еще хотя бы один --- последней.

    Пусть слово длины хотя бы 2. Тогда сопоставим первый символ первой скобке,
    второй скобке --- $\varepsilon$, третьей --- все остальные символы.

    Тогда можем построить очевидный автомат, который подходит под все три пункта
    (да, мне лень набирать три разных):
    \begin{myauto}
      \node[state]           (q_1)                {$q_1$};
      \node[state,initial]   (q_0) [left=of  q_1] {$q_0$};
      \node[state,accepting] (q_2) [right=of q_1] {$q_2$};
      \path[->]
        (q_0) edge              node [above] {$a, b$} (q_1)
        (q_1) edge              node [above] {$a, b$} (q_2)
        (q_2) edge [loop above] node [above] {$a, b$} ()
      ;
    \end{myauto}
    Про минимальность тоже понятно: $q_2$ принимает все слова,
    $q_1$ --- все непустые, $q_0$ --- все длины хотя бы 2.
  \item Построить регулярное выражение, распознающее тот же язык, что и автомат:
  \begin{myauto}
    \node[state]           (q_2)                {$q_2$};
    \node[state,initial]   (q_0) [left=of  q_2] {$q_0$};
    \node[state]           (q_1) [above=of q_2] {$q_1$};
    \node[state]           (q_3) [below=of q_2] {$q_3$};
    \node[state,accepting] (q_4) [right=of q_2] {$q_4$};

    \path[->] (q_0) edge [loop above] node [above] {$a, b, c$} ()
                    edge              node [above] {$a$}       (q_1)
                    edge              node [above] {$b$}       (q_2)
                    edge              node [above] {$c$}       (q_3)
              (q_1) edge [loop above] node [above] {$b, c$}    ()
                    edge              node [above] {$a$}       (q_4)
              (q_2) edge [loop above] node [above] {$a, c$}    ()
                    edge              node [above] {$b$}       (q_4)
              (q_3) edge [loop above] node [above] {$a, b$}    ()
                    edge              node [above] {$c$}       (q_4)
    ;
  \end{myauto}

    Решение:

    Кажется, что необязательно применять пошаговый алгоритм и рисовать миллион
    промежуточных автоматов... Методом внимательного взгляда можем
    получить выражение:
    \begin{equation*}
      {(a \mid b \mid c)}^* (a {(b \mid c)}^* a \mid b {(a \mid c)}^* b
      \mid c {(a \mid b)}^* c)
    \end{equation*}

    Если посмотреть еще внимательнее, становится ясно, что этот автомат (и выражение)
    принимают ровно те непустые слова, в которых последний символ не является
    единственным таким символом в слове.
  \item Определить, является ли автоматным язык $\{ \omega \omega^r \mid \omega \in \{ 0, 1 \}^* \}$. Если является --- построить автомат, иначе --- доказать.

    Решение:

    Вспомним, что автоматные языки --- это то же, что регулярные.

    Пусть наш язык регулярный, тогда зафиксируем $n$ из леммы о накачке.
    Рассмотрим слово $w = \omega \omega^r$, где $\omega = 0^{n-1}1$.
    Это слово длины $2n$, лежащее в языке.
    Возьмем разбиение из леммы $w = xyz, |xy| \le n, |y| \ge 1$.

    Если $y$ содержит единицу, то в слове $xyyz$ 3 единицы, и оно не может
    быть четным палиндромом.

    Иначе, $y$ состоит из одних нулей. Но тогда, если бы было верно
    $xyyz = \omega_0 \omega_0^r$, $\omega_0$ должно было бы содержать ровно одну
    единицу (т. к. в $xyyz$ их две), но тогда (поскольку единицы все еще стоят рядом)
    $\omega_0^r$ было бы короче, чем $\omega_0$.

    В обоих случаях получили противоречие, а значит, язык не регулярный.
  \item Определить, является ли автоматным язык $\{ u a a v \mid u, v \in \{ a, b \}^* , |u|_b \geq |v|_a \}$. Если является --- построить автомат, иначе --- доказать.

    Решение:

    Пусть язык регулярный, снова зафиксируем $n$ из леммы о накачке. Рассмотрим слово
    $w = b^n aa {(ba)}^n$. Это слово длины $2n + 2$, лежащее в языке. Возьмем разбиение из
    леммы $w = xyz, |xy| \le n, |y| \ge 1$. Заметим, что $y = b^p, p \ge 1$.
    Но тогда слово $xz = b^{n-p} aa {(ba)}^n$ также должно лежать в языке.
    Поскольку $aa$ входит в слово лишь один раз, возможен только случай
    $u = b^{n-p}, v = {(ba)}^n$, но тогда $n - p = |u|_b < |v|_a = n$.

    Получили противоречие, а значит, язык не регулярный.
\end{enumerate}

\newpage

\begin{center}
  \Large{Пример применения алгоритма минимизации}
\end{center}

\bigskip

Минимизируем данный автомат:

\begin{center}
  \begin{tikzpicture}[> = stealth,node distance=3cm, on grid]
    \node[state]           (q_2)                      {C};
    \node[state,initial]   (q_0) [above left=of q_2]  {A};
    \node[state]           (q_1) [below left=of q_2]  {B};
    \node[state]           (q_3) [right=of q_2]       {D};
    \node[state]           (q_4) [above right=of q_3] {E};
    \node[state,accepting] (q_5) [below right=of q_3] {F};
    \node[state,accepting] (q_6) [above right=of q_5] {G};

    \path[->] (q_0) edge [bend left=15]  node [right] {$1$} (q_1)
                    edge                 node [above] {$0$} (q_2)
              (q_1) edge [bend left=15]  node [left]  {$1$} (q_0)
                    edge                 node [below] {$0$} (q_2)
              (q_2) edge [bend right=15] node [below] {$1$} (q_3)
                    edge [bend left=15]  node [above] {$0$} (q_3)
              (q_3) edge                 node [below] {$1$} (q_5)
                    edge                 node [above] {$0$} (q_4)
              (q_4) edge                 node [above] {$1$} (q_6)
                    edge                 node [right] {$0$} (q_5)
              (q_5) edge [loop below]    node         {$1$} ()
                    edge [loop left]     node         {$0$} ()
              (q_6) edge                 node [below] {$1$} (q_5)
                    edge [loop right]    node         {$0$} ();
  \end{tikzpicture}
\end{center}

Автомат полный, в нем нет недостижимых вершин --- продолжаем.

Строим обратное $\delta$ отображение.

\begin{tabular}{c|c|c}
$\delta^{-1}$ & 0 & 1 \\ \hline
A & --- & B \\
B & --- & A \\
C & A B & --- \\
D & C & C \\
E & D & --- \\
F & E F & D F G \\
G & G & E
\end{tabular}

Отмечаем в таблице и добавляем в очередь пары состояний, различаемых словом $\varepsilon$: все пары, один элемент которых --- терминальное состояние, а второй --- не терминальное состояние. Для данного автомата это пары

$(A, F), (B, F), (C, F), (D, F), (E,F), (A, G), (B, G), (C, G), (D, G), (E, G)$

Дальше итерируем процесс определения неэквивалентных состояний, пока очередь не оказывается пуста.

$(A, F)$ не дает нам новых неэквивалентных пар. Для $(B, F)$ находится 2 пары: $(A, D), (A, G)$. Первая пара не отмечена в таблице --- отмечаем и добавляем в очередь. Вторая пара уже отмечена в таблице, значит, ничего делать не надо. Переходим к следующей паре из очереди. Итерируем дальше, пока очередь не опустошится.

Результирующая таблица (заполнен только треугольник, потому что остальное симметрично) и порядок добавления пар в очередь.

\begin{tabular}{c|cc|cc|cc|c}
& A & B & C & D & E & F & G \\ \hline
A &&&&&&& \\
B &&&&&&& \\ \hline
C & \checkmark & \checkmark &&&&& \\
D & \checkmark & \checkmark & \checkmark &&&& \\ \hline
E & \checkmark & \checkmark & \checkmark & \checkmark &&& \\
F & \checkmark & \checkmark & \checkmark & \checkmark & \checkmark && \\ \hline
G & \checkmark & \checkmark & \checkmark & \checkmark & \checkmark && \\
\end{tabular}

Очередь:

$
(A, F), (B, F), (C, F), (D, F), (E,F), (A, G), (B, G), (C, G), (D, G), (E, G),
$

$
(B, D), (A, D), (A, E), (B, E), (C, E), (C, D), (D, E), (A,C), (B, C))
$

В таблице выделились классы эквивалентных вершин: $\{A, B\}, \{C\}, \{D\}, \{E\}, \{F,G\}$. Остается только нарисовать результирующий автомат с вершинами-классами. Переходы добавляются тогда, когда из какого-нибудь состояния первого класса есть переход в какое-нибудь состояние второго класса. Минимизированный автомат:

\begin{center}
  \begin{tikzpicture}[> = stealth,node distance=3cm, on grid]
    \node[state,initial]   (q_01)                     {AB};
    \node[state]           (q_2)  [right=of q_01]      {C};
    \node[state]           (q_3)  [right=of q_2]       {D};
    \node[state]           (q_4)  [above right=of q_3] {E};
    \node[state,accepting] (q_56) [below right=of q_3] {FG};

    \path[->] (q_01) edge [loop above]    node [above] {$1$} ()
                     edge                 node [above] {$0$} (q_2)
              (q_2)  edge [bend right=15] node [below] {$1$} (q_3)
                     edge [bend left=15]  node [above] {$0$} (q_3)
              (q_3)  edge                 node [below] {$1$} (q_56)
                     edge                 node [above] {$0$} (q_4)
              (q_4)  edge [bend right=15] node [left]  {$1$} (q_56)
                     edge [bend left=15]  node [right] {$0$} (q_56)
              (q_56) edge [loop below]    node         {$1$} ()
                     edge [loop left]     node         {$0$} ();
  \end{tikzpicture}
\end{center}

\end{document}
